\documentclass[12pt]{article}


\usepackage[utf8]{inputenc}
\usepackage[T1]{fontenc}
\usepackage[ngerman]{babel}

\usepackage{xcolor}

 
\usepackage{tikz}
\usetikzlibrary{patterns}
\usepackage{amsmath}

\newcommand{\createlength}[2]{\newlength{#1}\setlength{#1}{#2}}
\createlength{\xmax}{5.7cm}
\createlength{\ymax}{5cm}
%
%% Koordinaten der Punkte
\createlength{\xa}{2.5cm}\createlength{\ya}{1cm}%
%
\createlength{\xb}{5.5cm}\createlength{\yb}{2cm}%
%
\createlength{\xc}{4.5cm}\createlength{\yc}{3.5cm}%
%
\createlength{\xd}{3.5cm}\createlength{\yd}{2cm}%
%
\createlength{\xe}{1.5cm}\createlength{\ye}{2.5cm}%
%%
%%
\begin{document}
\begin{tikzpicture}
%Grundstyle f\"ur alle Flaechen (20% Deckkraft)
\tikzstyle{FL}=[fill,draw,opacity=0.2] 

%%Style f\"ur einzelne Fl\"achen: positiv: blau, negativ: rot
\tikzstyle{F1}=[red,FL]
\tikzstyle{F2}=[blue,FL]
\tikzstyle{F3}=[blue,FL]
\tikzstyle{F4}=[blue,FL]
\tikzstyle{F5}=[red,FL]

%% das Grundpolygon, bitte die Definition  der x-,y- Koordinaten  ("=(x_i,y_i)") nur auf der(n) ersten Slides verwenden, danach ausblenden!
\draw[thick] (0,0) edge[->](\xmax,0);%x-achse
\draw[thick] (0,0) edge[->](0,\ymax);%y-achse
\draw (\xa,\ya) coordinate(A)   node[anchor=north]{$n_1$ $=(x_1,y_1)$}
 --     (\xb,\yb) coordinate(B)   node[anchor=west]{$n_2$ $=(x_2,y_2)$}
--      (\xc,\yc) coordinate(C)   node[anchor=south]{$n_3$ $=(x_3,y_3)$}
--      (\xd,\yd) coordinate(D)   node [anchor=north]{$n_4$ $=(x_4,y_4)$}
--      (\xe,\ye) coordinate(E)   node [anchor= south]  {$n_5$ $=(x_5,y_5)$} 
-- cycle;

%%Flaechen F1-F5
\path[F1] (\xa,0)--(A)--(B)--(\xb,0)--cycle;
\path ({(\xa+\xb)*0.5},0.2)node{F1};

\path[F2] (\xb,0)--(B)--(C)--(\xc,0)--cycle;
\path ({(\xb+\xc)*0.5},2.1)node{F2};

\path[F3] (\xc,0)--(C)--(D)--(\xd,0)--cycle;
\path ({(\xc+\xd)*0.5},2.1)node{F3};

\path[F4] (\xd,0)--(D)--(E)--(\xe,0)--cycle;
\path ({(\xd+\xe)*0.5},1.8)node{F4};

\path[F5] (\xe,0)--(E)--(A)--(\xa,0)--cycle;
\path ({(\xe+\xa)*0.5},0.2)node{F5};

%% finale Flaeche:
\path[pattern=horizontal lines,FL] (A)--(B)--(C)--(D)--(E)--cycle;

%Bsp. Trapez F1
\draw[red] (\xa,0.5)--node[midway,anchor=south]{$h$}(\xb,0.5);
\draw[red] (\xa,0)--node[midway,anchor=east]{$a$}(\xa,\ya);
\draw[red] (\xb,0)--node[midway,anchor=west]{$c$}(\xb,\yb);

%Bsp. Trapez F2
\draw[red] (\xb,0.5)--node[midway,anchor=south]{$h$}(\xc,0.5);
\draw[red] (\xb,0)--node[midway,anchor=west]{$a$}(\xb,\yb);
\draw[red] (\xc,0)--node[midway,anchor=east]{$c$}(\xc,\yc);


%Erl\"auterungen zu den verschiedenen Slides (bitte den Gesamten Inhalt auf verschiedene Slides aufteilen)
\path (\xmax,\ymax) ++(.5cm,1cm)node[anchor=north west, align=left]{
\begin{minipage}{5cm}
\small%% kleinere Schrift f\"ur mehr Inhalt
Trapezformel:
\[ A=\frac{1}{2}\sum_{i=1}^{n}  (x_i-x_{i+1})(y_i+y_{i+1}) \]

Idee: Blaue Fl. - rote Fl. \\ 
am Beispiel:\\

$F1$: Fl\"acheninhalt eines Trapezes: $A=\frac{1}{2} h*(a+c)$\\

$h$ \ldots H\"ohe $= x_2-x_1$\\
$a$ \ldots Grundseite $=y_1$\\ 
$c$ \ldots Basis $= y_2$ \\

$A=- \frac{1}{2} (x_1-x_2)*(y_1+y_2)$\\
($x_2 > x_1$: negativer Beitrag)\\


$F2$:  $A=\frac{1}{2} h*(a+c)$\\

$h$ \ldots H\"ohe $= x_2-x_3$\\
$a$ \ldots Grundseite $=y_2$\\ 
$c$ \ldots Basis $= y_3$ \\

$A= \frac{1}{2} (x_2-x_3)*(y_2+y_3)$\\
($x_3 < x_2$: positiver Beitrag)

\(-F1+F2 = \)
\begin{align*}
&&\frac{1}{2} (x_1-x_2)*&(y_1+y_2)\\
+&& \frac{1}{2} (x_2-x_3)*&(y_2+y_3)
\end{align*}
\[=\frac{1}{2}\sum_{i=1}^{2} (x_i-x_{i+1})(y_i+y_{i+1}) \]
\hfill $\cdots$ \hfill \hfill
\[ A=\frac{1}{2}\sum_{i=1}^{5} (x_i-x_{i+1})(y_i+y_{i+1})  \]
\end{minipage}
};%%end Erl\"auterungsknoten
\end{tikzpicture}
\end{document}
