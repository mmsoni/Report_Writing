\documentclass[	a4paper,      % Papiergröße
% 		parskip,      % Absatz statt Einzug zwischen Absätzen
	       ]{article}    % Dokumentklasse

\usepackage [utf8] {inputenc}
\usepackage[T1]{fontenc}


%%% Layout \"Anderungen f\"ur mehr Platz
\oddsidemargin0cm
\topmargin0cm
\textheight23.7cm	
\textwidth16cm
\headheight0cm
\headsep0cm
%
%% keine Einr\"uckung von Abs\"atzen
\setlength\parindent{0pt}

\usepackage[ngerman]{babel}
\usepackage{tikz}
\usepackage{xcolor}

\usetikzlibrary{backgrounds,scopes,shapes.geometric,shadows,decorations.shapes}


%%%  zu aufgabe g)
%%% color definitions
\colorlet{woodcol}{brown!70!black}
\colorlet{bodycol}{green!40!black}
\colorlet{waxcol}{yellow}
\colorlet{lightcol}{red}

\colorlet{dekoIIcolI}{red}
\colorlet{dekoIIcolII}{blue}
\colorlet{dekoIIcolIII}{green}

\colorlet{dekoIIIcolI}{red!80!white}
\colorlet{dekoIIIcolII}{blue!80!white}
%% end zu aufgabe g)


%%% zu Aufgabe c)
\newsavebox{\dekoOne}
\savebox{\dekoOne}{%
\begin{tikzpicture}
\draw(0,0)rectangle(2,1);
\draw(0,0)node[anchor=south west]{dekoOne};
\end{tikzpicture}%
}%end savebox
%%% end zu Aufgabe c)


%%
%% creates a help grid for given \emph{integer} coordinates 
%% (fractional coordinates do not work correctly)
%% #1: minimum x-coordinate, #2 maximum x-coordinate
%% #3  minimum y-coordinate, #4 maximum y-coordinate
%
\newcommand{\helpgrid}[4]{%
\draw[scale=0.1,line width=0.1pt,opacity=0.5]  (10*#1,10*#3) grid (10*#2,10*#4);
\draw[line width=1pt,opacity=0.2]  (#1,#3) grid (#2,#4);
\begin{scope}[every node/.style={font=\tiny,opacity=0.5}]
\foreach \x  in {#1,...,#2}
   \node[anchor=north] at (\x,#3){ \x };
\foreach \x  in {#1,...,#2}
   \node[anchor=south] at (\x,#4){ \x };
\foreach \y  in {#3,...,#4}
   \node[anchor=east] at (#1,\y){ \y };
\foreach \y in {#3,...,#4}
   \node[anchor=west] at (#2,\y){ \y };    
\end{scope}
}%

%%%
%
% zu Aufgabe b): eigener Befehl
%
\newcommand{\bodyshape}{
\path (-4,0) coordinate(A){} 
        (4,0) coordinate(B){}
        (0,4) coordinate(C){};
%%% todo: fill
%
%
}%end bodyshape
%%end zu Aufgabe b): eigener Befehl


\begin{document}

%% zu Aufgabe c)
%% todo delete this Demo-box:
\usebox{\dekoOne}


\begin{center}
%%% the main image
  \begin{tikzpicture}[ %% parameters for main image:
    mystar/.style={star, minimum size=1.5cm, star point ratio=2.5, shade, thick,
       line join=round, color=yellow!80!black, draw=red!20!black,
       top color=yellow!80!white, bottom color=yellow!60!black},
    myball/.style ={shade, ball color=#1, circular drop shadow={
     shadow xshift=0.05ex, shadow yshift=-0.1ex, fill=black, opacity=0.4}},
    starline/.style={decorate,decoration={shape backgrounds,shape=star, shape width=1.5mm, shape height=1.5mm,shape sep=2.2mm},
                  star points=6,draw=#1!50!black, fill=#1!50},
    ]%% end parameters for main image

%% zu aufgabe a):
\path[fill=woodcol](-1.5,0)--(1.5,0)to[bend left=25](0,3.5) to[bend left=25](-1.5,0);
%%end aufgabe a)  
  
  
%% koerper
%%zu aufgabe b):
\bodyshape
%erster scope TODO

%zweiter scope TODO
%%end aufgabe b)

%%% zu aufgabe c)
%TODO
%% end aufgabe c)

%% zu aufgabe d)
%TODO
%% end aufgabe d)
 
%% zu aufgabe a)
\helpgrid{-1}{1}{-2}{3}
%%end zu aufgabe a)
 \end{tikzpicture}
\end{center}
\end{document}