\documentclass[12pt]{book}


\usepackage[utf8]{inputenc}
\usepackage[T1]{fontenc}
\usepackage[english]{babel}
\usepackage{graphicx}
\usepackage{xcolor}
\usepackage{float}
%%
% #1 (optional) horizontale Gr\"o\ss{}e
% #2 vertikale Gr\"o\ss{}e
%%
\newcommand{\blueB}[2][\linewidth]{% 
{\color{blue}\rule{#1}{#2}}}
\newcommand{\greenB}[2][\linewidth]{%
{\color{green}\rule{#1}{#2}}}

\usepackage{hyperref}
\begin{document}

\chapter{Placing floats -- an exercise}
\section{The TeX Project}

The TeX project was started in 1978 by D Knuth, while revising the second volume of his Art of Computer Programming. When he got the
galleys back, he saw that the publisher had switched to a new digital typesetting system and was shocked at the poor quality (note Rule~1 in Sketch~\\ref{}).

% hier Gleitobjekt 1

He reasoned that because digital typesetting meant arranging 1's and 0's (ink and no ink) in the proper pattern, as a computer scientist
he should be able to do the job better. He originally estimated that this would take six months but ultimately it took nearly ten years.
He had to handle not only the challenges of routine typesetting such as right-justification and page formatting flexible enough to allow
for different output styles, but also the additional demands of academic publishing -- footnotes, floating figures and tables, etc. And,
beyond that, he had to tell the computer how to typset formulas and other technical materials (note Rule~2 in Figure~\ref{}).

% hier Gleitobjekt 2

A year after he began, Knuth was invited to present one of the principal lectures at the AMS's annual meeting. He spoke on his TeX work,
and also on Metafont (his system for developing fonts). He presented not only the roots of the typographical concepts, but also the
mathematical notions on which these two programs are based. TeX's popularity took off from there.

An important boost to that popularity came in 1985 with the introduction by L Lamport of LaTeX, a set of commands that allows interaction
with the system at a higher level than Knuth's original set (which is called Plain TeX) (note Rule~3 in Sketch~\ref{}).

% hier Gleitobjekt 3

\section{The TeX Project II, III}

The TeX project was started in 1978 by D Knuth, while revising the second volume of his Art of Computer Programming. When he got the
galleys back, he saw that the publisher had switched to a new digital typesetting system and was shocked at the poor quality (note Rule~4 in Figure~\ref{}).

% hier Gleitobjekt 4

He reasoned that because digital typesetting meant arranging 1's and 0's (ink and no ink) in the proper pattern, as a computer scientist
he should be able to do the job better. He originally estimated that this would take six months but ultimately it took nearly ten years.
He had to handle not only the challenges of routine typesetting such as right-justification and page formatting flexible enough to allow
for different output styles, but also the additional demands of academic publishing -- footnotes, floating figures and tables, etc. And,
beyond that, he had to tell the computer how to typset formulas and other technical materials (note Rule~5 in Sketch~\ref{}).

% hier Gleitobjekt 5

A year after he began, Knuth was invited to present one of the principal lectures at the AMS's annual meeting. He spoke on his TeX work,
and also on Metafont (his system for developing fonts). He presented not only the roots of the typographical concepts, but also the
mathematical notions on which these two programs are based. TeX's popularity took off from there.

An important boost to that popularity came in 1985 with the introduction by L Lamport of LaTeX, a set of commands that allows interaction
with the system at a higher level than Knuth's original set (which is called Plain TeX) (note Rule~6 in Figure~\ref{}).

% hier Gleitobjekt 6

The TeX project was started in 1978 by D Knuth, while revising the second volume of his Art of Computer Programming. When he got the
galleys back, he saw that the publisher had switched to a new digital typesetting system and was shocked at the poor quality (note Rule~7 in Sketch~\ref{}).

% hier Gleitobjekt 7

He reasoned that because digital typesetting meant arranging 1's and 0's (ink and no ink) in the proper pattern, as a computer scientist
he should be able to do the job better. He originally estimated that this would take six months but ultimately it took nearly ten years.
He had to handle not only the challenges of routine typesetting such as right-justification and page formatting flexible enough to allow
for different output styles, but also the additional demands of academic publishing -- footnotes, floating figures and tables, etc. And,
beyond that, he had to tell the computer how to typset formulas and other technical materials (note Rule~8 in Figure~\ref{}).

% hier Gleitobjekt 8

A year after he began, Knuth was invited to present one of the principal lectures at the AMS's annual meeting. He spoke on his TeX work,
and also on Metafont (his system for developing fonts). He presented not only the roots of the typographical concepts, but also the
mathematical notions on which these two programs are based. TeX's popularity took off from there.

An important boost to that popularity came in 1985 with the introduction by L Lamport of LaTeX, a set of commands that allows interaction
with the system at a higher level than Knuth's original set (which is called Plain TeX) (note Rule~9 in Sketch~\ref{}).

% hier Gleitobjekt 9

\section{The TeX Project IV,V}

The TeX project was started in 1978 by D Knuth, while revising the second volume of his Art of Computer Programming. When he got the
galleys back, he saw that the publisher had switched to a new digital typesetting system and was shocked at the poor quality (note Rule~10 in Figure~\ref{}).

% hier Gleitobjekt 10

He reasoned that because digital typesetting meant arranging 1's and 0's (ink and no ink) in the proper pattern, as a computer scientist
he should be able to do the job better. He originally estimated that this would take six months but ultimately it took nearly ten years.
He had to handle not only the challenges of routine typesetting such as right-justification and page formatting flexible enough to allow
for different output styles, but also the additional demands of academic publishing -- footnotes, floating figures and tables, etc. And,
beyond that, he had to tell the computer how to typset formulas and other technical materials (note Rule~11 in Sketch~\ref{}).

% hier Gleitobjekt 11

A year after he began, Knuth was invited to present one of the principal lectures at the AMS's annual meeting. He spoke on his TeX work,
and also on Metafont (his system for developing fonts). He presented not only the roots of the typographical concepts, but also the
mathematical notions on which these two programs are based. TeX's popularity took off from there.

An important boost to that popularity came in 1985 with the introduction by L Lamport of LaTeX, a set of commands that allows interaction
with the system at a higher level than Knuth's original set (which is called Plain TeX) (note Rule~12 in Figure\ref{}).

% hier Gleitobjekt 12

The TeX project was started in 1978 by D Knuth, while revising the second volume of his Art of Computer Programming. When he got the
galleys back, he saw that the publisher had switched to a new digital typesetting system and was shocked at the poor quality (note Rule~13 in Sketch~\ref{}).

% hier Gleitobjekt 13

He reasoned that because digital typesetting meant arranging 1's and 0's (ink and no ink) in the proper pattern, as a computer scientist
he should be able to do the job better. He originally estimated that this would take six months but ultimately it took nearly ten years.
He had to handle not only the challenges of routine typesetting such as right-justification and page formatting flexible enough to allow
for different output styles, but also the additional demands of academic publishing -- footnotes, floating figures and tables, etc. And,
beyond that, he had to tell the computer how to typset formulas and other technical materials (note Rule~14 in Figure~\ref{}).

% hier Gleitobjekt 14

A year after he began, Knuth was invited to present one of the principal lectures at the AMS's annual meeting. He spoke on his TeX work,
and also on Metafont (his system for developing fonts). He presented not only the roots of the typographical concepts, but also the
mathematical notions on which these two programs are based. TeX's popularity took off from there.

An important boost to that popularity came in 1985 with the introduction by L Lamport of LaTeX, a set of commands that allows interaction
with the system at a higher level than Knuth's original set (which is called Plain TeX) (note Rule~15 in Sketch~\ref{}).

% hier Gleitobjekt 15








% Gleitobjekt 1
\blueB[.3\linewidth]{2cm}

% Gleitobjekt 2
\greenB[.7\linewidth]{6cm}

% Gleitobjekt 3
\blueB[0.5\linewidth]{1cm}

% Gleitobjekt 4
\greenB[.6\linewidth]{8cm}

% Gleitobjekt 5
\blueB{3cm}

% Gleitobjekt 6
\greenB{2cm}

% Gleitobjekt 7
\blueB[0.7\linewidth]{2.5cm}

% Gleitobjekt 8
\greenB{6cm}

% Gleitobjekt 9
\blueB{1cm}

% Gleitobjekt 10
\greenB[0.9\linewidth]{8cm}

% Gleitobjekt 11
\blueB{6cm}

% Gleitobjekt 12
\greenB{9cm}

% Gleitobjekt 13
\blueB{9cm}

% Gleitobjekt 14
\greenB{5cm}

% Gleitobjekt 15
\blueB{5cm}
\end{document}